%% LyX 2.3.6 created this file.  For more info, see http://www.lyx.org/.
%% Do not edit unless you really know what you are doing.
\documentclass[polish]{article}
\usepackage[T1]{fontenc}
\usepackage{color}
\usepackage{babel}
\usepackage{url}
\usepackage[unicode=true,pdfusetitle,
 bookmarks=true,bookmarksnumbered=false,bookmarksopen=false,
 breaklinks=false,pdfborder={0 0 1},backref=false,colorlinks=true]
 {hyperref}

\makeatletter
%%%%%%%%%%%%%%%%%%%%%%%%%%%%%% User specified LaTeX commands.
\usepackage{indentfirst}

\makeatother

\begin{document}
\title{Metody ekstrakcji relacji w korpusie j\k{e}zyka polskiego}
\author{in\.{z}. Adam Lewandowski}
\maketitle

\section{Wprowadzenie}

\subsection{Opis dziedziny problemu}

Dynamiczny rozw\'{o}j In\.{z}ynierii Lingwistycznej w ostatniej dekadzie
zwi\k{e}kszy\l{} zainteresowanie os\'{o}b biznesowych wykorzystaniem
mo\.{z}liwo\'{s}ci modeli j\k{e}zyka w celu zwi\k{e}kszenia warto\'{s}ci
dodanej generowanej przez organizacj\k{e}. Oczekiwania wobec modeli
mog\k{a} by\'{c} wysokie z uwagi na imponuj\k{a}ce osi\k{a}gni\k{e}cia
transformer\'{o}w takich jak GPT czy BERT, kt\'{o}re przyczyniaj\k{a}
si\k{e} do ograniczania konieczno\'{s}ci konstrukcji w\l asnych rozwi\k{a}za\'{n},
wykorzystuj\k{a}c wiedz\k{e} o j\k{e}zyku wyindukowan\k{a} z wielu
korpus\'{o}w internetowych. 

Proces dopasowywania modeli j\k{e}zyka do w\l asnych zada\'{n} wymaga
dostarczenia dedykowanego zbioru danych, co bywa kosztowne przy pracy
wykorzystuj\k{a}cej modele j\k{e}zyk\'{o}w ma\l o popularnych. Przyczyn\k{a}
takiego stanu sytuacji jest najcz\k{e}\'{s}ciej trudno\'{s}\'{c} osi\k{a}gni\k{e}cia
zadowalaj\k{a}cych wynik\'{o}w modeli, kt\'{o}re nie posiadaj\k{a}
zdolno\'{s}ci prawid\l owej analizy semantycznej lub syntaktycznej
danego j\k{e}zyka, jak r\'{o}wnie\.{z} brak pe\l nego zaanga\.{z}owania
spo\l eczno\'{s}ci w proces rozwoju narz\k{e}dzi i zbior\'{o}w danych
w tym j\k{e}zyku. 

Uproszczenie analizy j\k{e}zyka pozwala na stworzenie rozwi\k{a}zania
problemu, jednak\.{z}e ogranicza to mo\.{z}liwo\'{s}ci stosowania
modeli j\k{e}zyka. Rozwi\k{a}zaniem po\'{s}rednim jest wykorzystanie
algorytm\'{o}w Ekstrakcji Informacji w celu wyodr\k{e}bnienia encji
i \l \k{a}cz\k{a}cych ich relacji, kt\'{o}re posiadaj\k{a} mo\.{z}liwo\'{s}\'{c}
wykonywania analizy semantycznej s\l \'{o}w ograniczaj\k{a}c przy
tym wp\l yw analizy syntaktycznej w procesie realizacji zadania. Proces
ekstrakcji encji, nazywany Wykrywaniem Encji, okre\'{s}la semantyk\k{e}
wybranych s\l \'{o}w, jednak\.{z}e jest ona niewystarczaj\k{a}ca do
pe\l nego zrozumienia znaczenia dokumentu. 

Encje okre\'{s}laj\k{a} uniwersalny kontekst statyczny dla s\l owa,
natomiast zdanie mo\.{z}e by\'{c} sekwencj\k{a} wielu r\'{o}\.{z}nych
permutacji encji, co wymusza zmian\k{e} kontekstu danej encji. Detekcja
kontekstu encji w sekwencji mo\.{z}e by\'{c} zadaniem nietrywialnym,
co jest g\l \'{o}wn\k{a} problematyk\k{a} dziedziny \emph{Word-sense
disambiguation. }Skuteczniejszym rozwi\k{a}zaniem problemu wieloznaczno\'{s}ci
encji jest wykorzystanie metod Ekstrakcji Relacji, kt\'{o}re pozwalaj\k{a}
okre\'{s}li\'{c} semantyk\k{e} mi\k{e}dzy encjami. 

Istotnym problemem w wykorzystaniu metod Ekstrakcji Relacji w korpusie
j\k{e}zyka polskiego jest niska ilo\'{s}\'{c} prac naukowych, narz\k{e}dzi
oraz zbior\'{o}w danych po\'{s}wi\k{e}conych temu j\k{e}zykowi. Implikuje
to hipotez\k{e} badawcz\k{a} dotycz\k{a}c\k{a} jako\'{s}ci wyekstraktowanych
relacji przy wykorzystaniu r\'{o}\.{z}nych metod, g\l \'{o}wnie dedykowanych
j\k{e}zykowi angielskiemu oraz detekcji potencjalnych problem\'{o}w
w przypadku ich u\.{z}ycia.

\subsection{Cel i zakres pracy}

Celem pracy jest analiza i ocena jako\'{s}ci metod ekstrakcji relacji
wykorzystywanych w dokumentach zapisanych w j\k{e}zyku polskim oraz
wyszczeg\'{o}lnienie istnienia potencjalnych problem\'{o}w przy ich
zastosowaniu. 

Realizacja celu wymaga starannego zaplanowania kolejno\'{s}ci realizacji
poszczeg\'{o}lnych zada\'{n}, ze wzgl\k{e}du na trudno\'{s}\'{c} w
uzyskaniu materia\l \'{o}w dedykowanych j\k{e}zykowi polskiemu, a
tak\.{z}e badawczego charakteru pracy, cechuj\k{a}cego si\k{e} potencjaln\k{a}
innowacyjno\'{s}ci\k{a}, mog\k{a}c\k{a} poszerzy\'{c} zainteresowanie
zagranicznych inwestor\'{o}w w produkcj\k{e} us\l ug wykorzystuj\k{a}cych
j\k{e}zyk polski. 

Zdecydowano si\k{e} na rozpocz\k{e}cie prac od wprowadzenia czytelnika
do opisu problem\'{o}w, zdefiniowania zada\'{n}, kt\'{o}re umo\.{z}liwiaj\k{a}
nakierowanie na potencjalne rozwi\k{a}zania wraz z opisem dost\k{e}pnych
metod, miar i zbior\'{o}w danych, opisane w rozdziale \nameref{sec:Opis-teoretyczny}.
Nast\k{e}pnie dokonano opisania metodologii badawczej przyj\k{e}tej
w niniejszej pracy w rozdziale \nameref{sec:Opis-eksperyment=0000F3w}.
W rozdziale \nameref{sec:Wyniki-eksperyment=0000F3w} zamieszczone
zosta\l y por\'{o}wnania jako\'{s}ci metod Ekstrakcji Relacji w formie
metryk oraz osobistej mierze jako\'{s}ci stworzonych relacji. Naznaczenie
potencjalnych problem\'{o}w wraz z wnioskami zosta\l o umieszczone
w rozdziale \nameref{sec:Analiza-wynik=0000F3w}. Rozdzia\l{} \nameref{sec:Podsumowanie}
w zwi\k{e}z\l y spos\'{o}b okre\'{s}la ca\l okszta\l t pracy, natomiast
w rozdziale \nameref{sec:Mo=00017Cliwo=00015Bci-rozszerzenia-pracy}
wyznaczone zosta\l y kierunki dalszego rozwoju prac.

Lista zada\'{n} zrealizowana w procesie realizacji pracy dyplomowej
jest nast\k{e}puj\k{a}ca:
\begin{itemize}
\item zapoznanie si\k{e} z tematyk\k{a} dziedziny Ekstrakcji Relacji, dost\k{e}pnych
zbior\'{o}w danych, metryk, metod,
\item wyszukanie i selekcja artyku\l \'{o}w opublikowanych z ostatnich lat
opisuj\k{a}cych metody ekstrakcji relacji,
\item wyszukanie i selekcja dost\k{e}pnych zbior\'{o}w danych,
\item wyb\'{o}r odpowiednich miar por\'{o}wnawczych,
\item wykorzystanie kodu \'{z}r\'{o}d\l owego modeli w celi przeprowadzenia
ewaluacji na zbiorze danych,
\item analiza wynik\'{o}w,
\item detekcja ewentualnych problem\'{o}w w korpusie j\k{e}zyka polskiego.
\end{itemize}

\section{Powi\k{a}zane prace}

Inspiracj\k{a} do napisania poni\.{z}szej pracy by\l a praca doktorska
autorstwa Aleksandra Smywi\'{n}ski-Pohla, ,,Automatyczna ekstrakcja
relacji semantycznych z tekst\'{o}w w j\k{e}zyku polskim'' \cite{POHL 2015},
w kt\'{o}rej autor poruszy\l{} temat aktualnego stanu wiedzy o dziedzinie
Ekstrakcji Relacji, wyst\k{e}puj\k{a}ce problemy w implementacji metod
Ekstrakcji Relacji, mo\.{z}liwe zastosowania oraz adaptacj\k{e} technik
do dokument\'{o}w w j\k{e}zyku polskim. 

Praca Mayank Kejriwal i innych, ,,Knowledge Graphs Fundamentals''
\cite{MAYA 2021} pozwala wprowadzi\'{c} czytelnika w tematyk\k{e}
Graf\'{o}w Wiedzy, kt\'{o}ra jest powi\k{a}zana z zadaniem Ekstrakcji
Relacji. Autorzy przedstawili aktualny stan wiedzy w dziedzinie Ekstrakcji
Relacji, co stanowi\l o inspiracj\k{e} do napisania rozdzia\l u teoretycznego
poni\.{z}szej pracy.

Czytelnik zainteresowany wykorzystaniem neuronowych modeli g\l \k{e}bokich
mo\.{z}e si\k{e}gn\k{a}\'{c} po artyku\l{} Shantanu Kumar, ,,A Survey
of Deep Learning Methods for Relation Extraction'' \cite{SHAN 2017}.
Opisano w nim wykorzystanie modeli konwolucyjnych w zadaniu Ekstrakcji
Relacji oraz por\'{o}wnano trywialne w implementacji sieci konwolucyjne
z zaawansowanymi wykorzystuj\k{a}cymi mechanizm uwa\.{z}no\'{s}ci.

Aktualny przegl\k{a}d prac nad dziedzin\k{a} Ekstrakcji Relacji mo\.{z}na
znale\'{s}\'{c} w artykule Sachin Pawar i innych, ,,Relation Extraction
: A Survey'' \cite{SACH 2017}.

\section{Opis teoretyczny\label{sec:Opis-teoretyczny}}

\subsection{Ekstrakcja informacji}

\subsubsection{Opis poj\k{e}cia}

Jak zdefiniowano w pracy \cite{POHL 2015}, cyt. ,,Ekstrakcja informacji
{[}ang. information extraction{]} jest procesem nadawania znaczenia
(interpretacji), w kt\'{o}rym przechodzi si\k{e}, od opisu danych
w terminach meta-j\k{e}zyka, do opisu w terminach j\k{e}zyka przedmiotowego,
dzieki czemu uzyskane informacje mog\k{a} by\'{c} bezpo\'{s}rednio
wykorzystane w zadaniach przetwarzania informacji. Ekstrakcja informacji
zwykle ogranicza si\k{e} do interpretowania pewnego podzbioru dostepnych
informacji, istotnych z punktu widzenia realizowanego zadania.'' 

Proces przypisywania znacze\'{n} nie modeluje semantyki sekwencji
token\'{o}w, ale poszczeg\'{o}lnych wyra\.{z}e\'{n} frazowych oraz
istniej\k{a}cych relacji pomi\k{e}dzy nimi, przydatnymi dla realizacji
okre\'{s}lonego zadania. Wyra\.{z}enia takie s\k{a} zdefiniowane w
pewnej ontologii dziedzinowej, dlatego wyra\.{z}enia frazowe mog\k{a}
posiada\'{c} przypisan\k{a} p\l ytk\k{a} semantyke. Celem ekstrakcji
informacji nie jest pe\l ne zrozumienie semantyki tekstu, ale wystarczaj\k{a}cych
jej fragment\'{o}w pozwalaj\k{a}cych zrealizowanie okre\'{s}lonego
zadania. 

\subsubsection{Obszary zastosowa\'{n}}

Istnieje wiele obszar\'{o}w zastosowa\'{n} algorytm\'{o}w ekstrakcji
informacji, przyk\l adowo:
\begin{itemize}
\item wyszukiwanie informacji poprzez wykorzystywanie cz\k{a}stkowych informacji
w dokumentach,
\item rozpoznawanie slot\'{o}w w systemach dialogowych i generowanie odpowiedzi,
\item wype\l nianie baz danych informacjami,
\item detekcja s\l \'{o}w kluczowych,
\item generowanie streszcze\'{n} dokument\'{o}w,
\item wype\l nianie szablon\'{o}w.
\end{itemize}

\subsubsection{Potencjalne problemy}

Wykorzystanie algorytm\'{o}w ekstrakcji informacji wymaga przeanalizowania
wyst\k{e}puj\k{a}cych w nich problem\'{o}w. 

Cz\k{e}sto wyst\k{e}puj\k{a}cym problemem jest wieloznaczno\'{s}\'{c}
(polisemia) wyra\.{z}e\'{n} frazowych, na kt\'{o}r\k{a} sk\l adaj\k{a}
si\k{e} intepretacje wyra\.{z}e\'{n} w s\l ownikach jak i interpretacje
zale\.{z}ne od kontekstu w kt\'{o}rym wyst\k{e}puj\k{a}. S\l ownikowa
definicja s\l owa ,,zamek'' w \emph{S\l owniku J\k{e}zyka Polskiego}\cite{SJP 2022}
przedstawia kilka mo\.{z}liwo\'{s}ci intepretacji bezkontekstowej
s\l owa - jako ,,urz\k{a}dzenie do zamykania drzwi'', ,,urz\k{a}dzenie
do \l \k{a}czenia lub zabezpieczania w ustalonym po\l o\.{z}eniu element\'{o}w
maszyny'' i ,,okaza\l a budowla mieszkalno-obronna''. W s\l owniku
\emph{Wikis\l ownik} \cite{WSLO 2022} istnieje kilka dodatkowych
znacze\'{n} tj. ,,zak\l ad penitencjarny lub jego wydzielona cz\k{e}\'{s}\'{c}
o zaostrzonym rygorze'', ,,w hokeju: zamkni\k{e}cie przeciwnika
w tercji, gdy dru\.{z}yna atakuj\k{a}ca rozgrywa kr\k{a}\.{z}ek w
tercji przeciwnika nie pozwalaj\k{a}c mu wyj\'{s}\'{c} poza niebiesk\k{a}
lini\k{e}''. Jak mo\.{z}na zauwa\.{z}y\'{c} istnieje pewna rozbie\.{z}no\'{s}\'{c}
intepretacji pomi\k{e}dzy s\l ownikami jak i wiele mo\.{z}liwo\'{s}ci
interpretacji danego wyra\.{z}enia. Co wi\k{e}cej, wyra\.{z}enia wsp\'{o}\l wyst\k{e}puj\k{a}ce
mog\k{a} modyfikowa\'{c} znaczenie danego wyra\.{z}enia frazowego
jak cho\'{c}by w zdaniu ,,Odwiedzi\l em zamek w Malborku.'' wyra\.{z}enia
,,odwiedzi\l em'' oraz ,,w Malborku'' pozwalaj\k{a} intepretowa\'{c}
s\l owo ,,zamek'' w kontek\'{s}cie fortyfikacji. W zdaniu ,,Kupi\l em
zamek do drzwi.'' modyfikatorem semantyki s\l owa jest wyra\.{z}enie
,,do drzwi'', kt\'{o}re zmienia jego znaczenie na urz\k{a}dzenie
do zamykania drzwi. Pomimo prostszej analizy semantycznej tekstu problem
wieloznaczno\'{s}ci ma znacz\k{a}cy wp\l yw na jako\'{s}\'{c} realizacji
pewnych zada\'{n}.

Powi\k{a}zanym problemem s\k{a} wyra\.{z}enia metaforyczne, tworz\k{a}ce
nowe znaczenia intepretacyjne wyra\.{z}e\'{n}. J\k{e}zyk ewoluowa\l{}
na przestrzeni wiek\'{o}w, jednak\.{z}e proces ten nie jest zako\'{n}czony,
u\.{z}ytkownicy j\k{e}zyka posiadaj\k{a} mo\.{z}liwo\'{s}\'{c} tworzenia
nowych poj\k{e}\'{c} wykorzystuj\k{a}c z\l \k{a}czenie ju\.{z} istniej\k{a}cych.
Prowadzi\'{c} to mo\.{z}e do sytuacji b\l \k{e}dnej intepretacji wyra\.{z}e\'{n}
ze wzgl\k{e}du na dezaktualizacj\k{e} wiedzy o ich z\l \k{a}czeniach.
Prawid\l owa intepretacja takich wyra\.{z}e\'{n} jak ,,pan m\l ody''
czy ,,urwanie g\l owy'' musi by\'{c} poprzedzona procesem aktualizacji
zapami\k{e}tanej semantyki wyra\.{z}e\'{n} frazowych. 

Nazwy w\l asne mog\k{a} powodowa\'{c} wiele niepo\.{z}\k{a}danych
problem\'{o}w zwi\k{a}zanych z ich liczno\'{s}ci\k{a} czy odmian\k{a}.
Ze wzgl\k{e}du na mnogo\'{s}\'{c} zapisu nazw w\l asnych w procesie
konstrukcji s\l ownik\'{o}w niemo\.{z}liwe jest uwzgl\k{e}dnienie
wszystkich ich odmian. Co wi\k{e}cej, istnieje r\'{o}wnie\.{z} wiele
mo\.{z}liwych odmian danego s\l owa. Istniej\k{a} co prawda algorytmy
pozwalaj\k{a}ce transformowa\'{c} wyra\.{z}enia do prostszych form
zachowuj\k{a}cych wi\k{e}ksz\k{a} cz\k{e}\'{s}\'{c} semantyki takie
jak lematyzacja czy stemming, nie rozwi\k{a}zuje to jednak g\l \'{o}wnego
problemu istnienia r\'{o}wnoleg\l ych form zapisu nazw w\l asnych.

Ostatnim z poruszanych problem\'{o}w jest niezgodno\'{s}\'{c} pomi\k{e}dzy
anotatorami w procesie anotacji zbior\'{o}w danych. R\'{o}\.{z}nice
mog\k{a} by\'{c} zauwa\.{z}alne przy kategoryzacji, wyborze odpowiednich
zakres\'{o}w frazy czy okre\'{s}lenia relacji pomi\k{e}dzy frazami.
Mo\.{z}e to wynika\'{c} z osobistych preferencji anotator\'{o}w, b\l \k{e}d\'{o}w
selekcji p\'{o}l czy nieprawid\l owej analizie tekstu. Mo\.{z}na okre\'{s}li\'{c}
konsensus w wyborze anotacji, nale\.{z}y jednak bra\'{c} pod uwag\k{e}
mo\.{z}liwo\'{s}\'{c} wyst\k{a}pienia problemu biasu w zbiorze danych.

\subsubsection{Zadania w procesie ekstrakcji informacji}

Dziedzina ekstrakcji informacji realizuje nast\k{e}puj\k{a}ce typy
zada\'{n}:
\begin{enumerate}
\item rozpoznawanie jednostek nazewniczych (ang. named entity recognition),
\item ekstrakcja relacji (ang. relation extraction),
\item rozpoznawanie wyra\.{z}e\'{n} wsp\'{o}lodnoszacych sie, (ang. coreference
resolution),
\item rozpoznawanie wyra\.{z}e\'{n} temporalnych (ang. temporal expression
recognition),
\item ekstrakcja zdarze\'{n} (ang. event extraction),
\item wypelnianie szablon\'{o}w (ang. template filling).
\end{enumerate}

\subsection{Rozpoznawanie jednostek nazewniczych}

\subsubsection{Opis poj\k{e}cia}

Rozpoznawanie jednostek nazewniczych (ang. Named Entity Recognition)
\cite{MOHA 2018} polega na wyszukaniu w dokumencie wyra\.{z}e\'{n},
kt\'{o}re spe\l niaj\k{a} za\l o\.{z}enia pozwalaj\k{a}ce na przypisanie
ich do kategorii semantycznej. Za\l o\.{z}eniem mo\.{z}e by\'{c} spe\l nialno\'{s}\'{c}
wyra\.{z}e\'{n} regularnych lub wyst\k{e}powanie wyra\.{z}enia w wybranej
ontologii. Kategori\k{a} semantyczn\k{a} nazywamy zbi\'{o}r obiekt\'{o}w
posiadaj\k{a}cych identyczn\k{a} lub podobn\k{a} semantyk\k{e}. Przyk\l adem
mo\.{z}e by\'{c} rozpoznanie wyra\.{z}e\'{n} reprezentuj\k{a}cych
pewne osoby, organizacje, lokalizacje, liczby czy numery rejestracyjne.
Dost\k{e}pne narz\k{e}dzia zazwyczaj posiadaj\k{a} predefiniowane
zbiory znacznik\'{o}w takie jak \textbf{ORG} (organizacje), \textbf{PER}
(osoby), \textbf{LOC} (lokalizacje). \cite{SPAC 2022} Istnieje mo\.{z}liwo\'{s}\'{c}
zdefiniowania w\l asnych zbior\'{o}w znacznik\'{o}w oraz rozszerzenie
ju\.{z} istniej\k{a}cych.

\subsubsection{Obszary zastosowa\'{n}}

Istnieje wiele mo\.{z}liwych obszar\'{o}w zastosowa\'{n} metod rozpoznawania
jednostek nazewniczych, w oparciu o przyk\l adowe przypadki u\.{z}ycia
\cite{SHAS 2018} mo\.{z}na wyr\'{o}\.{z}ni\'{c}: 
\begin{itemize}
\item kategoryzacj\k{e} dokument\'{o}w wykorzystuj\k{a}c\k{a} ekstrakcj\k{e}
s\l \'{o}w kluczowych,
\item wyszukiwanie relewantnych dokument\'{o}w zawieraj\k{a}cych wyst\k{e}puj\k{a}ce
jednostki nazewnicze w zapytaniu,
\item rekomendacj\k{e} dla czytelnik\'{o}w w oparciu o wyst\k{e}powanie
podobnych jednostek nazewniczych,
\item przy\'{s}pieszenie procesu udzielenia wsparcia klientowi.
\end{itemize}

\subsection{Ekstrakcja relacji}

\subsubsection{Metody}

Na czym polega, co na wejsciu, wyjsciu itp co oznacza
\begin{itemize}
\item EType oraz EType+
\end{itemize}
\textbf{EType }jest metod\k{a} indukuj\k{a}c\k{a} relacje z konkatenacji
dw\'{o}ch typ\'{o}w encji nazwanych, $t_{e_{head}}$oraz $t_{e_{tail}}$.
Wektory zero-jedynkowe (ang. one-hot-encoding) dla $t_{e_{head}}$oraz
$t_{e_{tail}}$podawane s\k{a} na wej\'{s}ciu jednowarstwowej neuronowej
sieci g\l \k{e}bokiej, kt\'{o}ra okre\'{s}la prawdopodobie\'{n}stwo
\l \k{a}cz\k{a}cych je relacji. Najbardziej prawdopodobne z\l \k{a}czenia
s\k{a} wyekstraktowane. Model neuronowy jest uczony w spos\'{o}b nienadzorowany.
(na jakiej podst. prawd? co jest na wej\'{s}ciu? framework opisywania)

\subsubsection{Zbiory danych}
\begin{itemize}
\item NYT-FB
\item TACRED (The TAC Relation Extraction Dataset) 
\item SemEval-2010 Task 8
\item FewRel
\item WMORC
\item NeuralOIE
\item ClausIE
\item Raw Web/Parallel En-Sp
\item Wiki80
\item NYT10
\item BSNLP 2019 Shared Task
\item ACE 2003/2004/2005/2007
\end{itemize}

\subsubsection{Miary por\'{o}wnawcze}
\begin{itemize}
\item Accuracy - how close or far off a given set of measurements (observations
or readings) are to their true value {[}zrobic swoimi slowami!!!{]}
\item F-score - The F1 score is the harmonic mean of the precision and recall.
The more generic \{\textbackslash displaystyle F\_\{\textbackslash beta
\}\}F\_\{\textbackslash beta \} score applies additional weights,
valuing one of precision or recall more than the other. The highest
possible value of an F-score is 1.0, indicating perfect precision
and recall, and the lowest possible value is 0, if either the precision
or the recall is zero {[}zrobic swoimi slowami!!!{]}
\item AUC
\item Precision - how close or dispersed the measurements are to each other
{[}zrobic swoimi slowami!!!{]}
\item Recall
\item RPM
\item REM
\item SM
\item V
\item ARI
\item $B^{3}$
\item P@N
\item Silhouette
\end{itemize}

\section{Opis eksperyment\'{o}w\label{sec:Opis-eksperyment=0000F3w}}

\subsection{Zbiory danych}

Ze wzgl\k{e}du na nisk\k{a} dost\k{e}pno\'{s}\'{c} dobrej jako\'{s}ci
zbior\'{o}w danych polskoj\k{e}zycznych dedykowanych zadaniu ekstrakcji
relacji zdecydowano si\k{e} na wykonanie procesu translacji angielskoj\k{e}zycznych
zbior\'{o}w danych. W tym celu wykorzystano darmowy i publicznie dost\k{e}pny
translator Google. Rozwa\.{z}ano r\'{o}wnie\.{z} wykorzystanie konkurencyjnego
produktu DeepL, kt\'{o}ry cechuje si\k{e} mo\.{z}liwo\'{s}ci\k{a}
wyboru stylu tekstu, jednak ze wzgl\k{e}du na ograniczenia w ilo\'{s}ci
znak\'{o}w do przet\l umaczenia zrezygnowano z tego pomys\l u.

\subsubsection{SemEval}

\textbf{SemEval-2010 Task 8} \cite{HEND 2010} skupia si\k{e} na zadaniu
klasyfikacji wieloetykietowanej.

Dane zosta\l y pobrane z serwisu Kaggle. \cite{SEME 2020} 

\subsubsection{FewRel}

\textbf{FewRel} \cite{HAN 2018} jest zbiorem danych dedykowanym zadaniu
ekstrakcji relacji w oparciu o technik\k{e} uczenia kilku-strza\l owego.
Sk\l ada si\k{e} z relacji wyekstraktowanych z r\'{o}\.{z}nych domen
dziedzinowych znajduj\k{a}cych si\k{e} w korpusie j\k{e}zyka formalnego.
Dane zosta\l y automatycznie wyekstraktowane z angielskoj\k{e}zycznej
strony Wikipedii przy wykorzystaniu relacji znajduj\k{a}cych si\k{e}
w ontologicznej bazie wiedzy Wikidata oraz poprawione manualnie przez
anotator\'{o}w.

Zbi\'{o}r danych mo\.{z}na uzyska\'{c} w dw\'{o}ch wersjach: \cite{FEWR 2018} 
\begin{itemize}
\item wersji 1.0, zawier\k{a}cej 64 relacje w zbiorze treningowym oraz 16
relacji w zbiorze walidacyjnym 
\item wersji 2.0, poprawiaj\k{a}cej aspekt zbytniej r\'{o}\.{z}norodno\'{s}ci
domen dziedzinowych oraz problemy wynikaj\k{a}ce ze stosowania uczenia
kilku-strza\l owego
\end{itemize}
W poni\.{z}szej pracy zdecydowano si\k{e} na u\.{z}ycie zbioru danych
w wersji 1.0 z uwagi na jego prostot\k{e} w adaptacji do przet\l umaczenia
na j\k{e}zyk polski. Ze wzgl\k{e}du na r\'{o}\.{z}norodno\'{s}\'{c}
analizowanych technik ekstrakcji relacji zdecydowano si\k{e} ograniczy\'{c}
zbi\'{o}r danych do zbioru treningowego, poniewa\.{z} relacje wyst\k{e}puj\k{a}ce
w obu podzbiorach nie posiadaj\k{a} cz\k{e}\'{s}ci wsp\'{o}lnej.

Relacje oraz encje opisane w zbiorze danych s\k{a} zapisane w postaci
identyfikatora obiektu wyst\k{e}puj\k{a}cego w bazie ontologicznej
Wikidata. W celu uzyskania semantycznego opisu nale\.{z}a\l o odnale\'{s}\'{c}
referencj\k{e} do obiektu oraz odczyta\'{c} odpowiednie informacje.
Kolejno\'{s}\'{c} wyst\k{e}powania encji w relacji jest znacz\k{a}ca.

\subsection{Miary por\'{o}wnawcze}

Dla zada\'{n} uczenia nadzorowanego i p\'{o}\l -nadzorowanego postanowiono
wykorzysta\'{c} metryki precision, recall oraz F-score, natomiast
dla zada\'{n} uczenia nienadzorowanego metryk\k{e} Silhouette.

\section{Wyniki eksperyment\'{o}w\label{sec:Wyniki-eksperyment=0000F3w}}

\section{Analiza wynik\'{o}w\label{sec:Analiza-wynik=0000F3w}}

\section{Podsumowanie\label{sec:Podsumowanie}}

\section{Mo\.{z}liwo\'{s}ci rozszerzenia pracy\label{sec:Mo=00017Cliwo=00015Bci-rozszerzenia-pracy}}

\section*{S\l ownik}

\section*{Lista wymboli}

\section*{Podzi\k{e}kowanie}
\begin{thebibliography}{TRAN 2020}
\bibitem[TRAN 2020]{TRAN 2020}Thy Thy Tran et al., Revisiting Unsupervised
Relation Extraction, Association for Computational Linguistics, 2020,
s. 7498--7505, \url{https://aclanthology.org/2020.acl-main.669/},
dost\k{e}p wolny dnia 7 czerwca 2022r.

\bibitem[POHL 2015]{POHL 2015}Aleksander Smywi\'{n}ski-Pohl, Automatyczna
ekstrakcja relacji semantycznych z tekst\'{o}w w j\k{e}zyku polskim,
niepublikowana praca doktorska, Akademia G\'{o}rniczo-Hutnicza im.
Stanislawa Staszica w Krakowie, 2015.

\bibitem[MAYA 2021]{MAYA 2021}Mayank Kejriwal et al., Knowledge Graphs
Fundamentals, Techniques, and Applications, MIT Press, 2021, s. 125-147,
\url{https://mitpress.mit.edu/books/knowledge-graphs}, dost\k{e}p
zastrze\.{z}ony dnia 2022r.

\bibitem[HEND 2010]{HEND 2010}Iris Hendrickx et al., SemEval-2010
Task 8: Multi-Way Classification of Semantic Relations between Pairs
of Nominals, Association for Computational Linguistics, Uppsala, Szwecja,
2010, s. 33--38, \url{https://aclanthology.org/S10-1006/}, dost\k{e}p
wolny dnia 7 czerwca 2022r.

\bibitem[HAN 2018]{HAN 2018}Xu Han et al., FewRel: A Large-Scale
Supervised Few-Shot Relation Classification Dataset with State-of-the-Art
Evaluation, Association for Computational Linguistics, Bruksela, Belgia,
2018, s. 4803--4809, \url{https://aclanthology.org/D18-1514/}, dost\k{e}p
wolny dnia 7 czerwca 2022r.

\bibitem[SHAN 2017]{SHAN 2017}Shantanu Kumar, A Survey of Deep Learning
Methods for Relation Extraction, 2017, \url{https://www.researchgate.net/publication/316859287_A_Survey_of_Deep_Learning_Methods_for_Relation_Extraction},
dost\k{e}p wolny dnia 8 czerwca 2022r.

\bibitem[SACH 2017]{SACH 2017}Sachin Pawar et al., Relation Extraction
: A Survey, 2017, \url{https://arxiv.org/abs/1712.05191}, dost\k{e}p
wolny dnia 8 czerwca 2022r.

\bibitem[SEME 2020]{SEME 2020}Adres do pobrania zbioru danych SemEval,
\url{https://www.kaggle.com/datasets/drtoshi/semeval2010-task-8-dataset?resource=download},
dost\k{e}p wolny dnia 7 czerwca 2022r.

\bibitem[FEWR 2018]{FEWR 2018}Adres do pobrania zbioru danych FewRel,
\url{https://www.zhuhao.me/fewrel/}, dost\k{e}p wolny dnia 7 czerwca
2022r.

\bibitem[SJP 2022]{SJP 2022}S\l ownik J\k{e}zyka Polskiego, definicje
s\l owa zamek, \url{https://sjp.pwn.pl/slowniki/zamek.html}, dost\k{e}p
wolny dnia 12 czerwca 2022r.

\bibitem[WSLO 2022]{WSLO 2022}Wikis\l ownik, definicje s\l owa zamek,
\url{https://pl.wiktionary.org/wiki/zamek}, dost\k{e}p wolny dnia
12 czerwca 2022r.

\bibitem[SPAC 2022]{SPAC 2022}Opis jednostek nazewniczych w narz\k{e}dziu
SpaCy, \url{https://spacy.io/usage/linguistic-features#named-entities},
dost\k{e}p wolny dnia 12 czerwca 2022r.

\bibitem[SHAS 2018]{SHAS 2018}Shashank Gupta, Named Entity Recognition:
Applications and Use Cases, Towards Data Science, 2018, \url{https://towardsdatascience.com/named-entity-recognition-applications-and-use-cases-acdbf57d595e},
dost\k{e}p limitowany dnia 12 czerwca 2022r.

\bibitem[MOHA 2018]{MOHA 2018}Mohan Gupta, A Review of Named Entity
Recognition (NER) Using Automatic Summarization of Resumes, Towards
Data Science, 2018, \url{https://towardsdatascience.com/a-review-of-named-entity-recognition-ner-using-automatic-summarization-of-resumes-5248a75de175},
dost\k{e}p limitowany dnia 12 czerwca 2022r.
\end{thebibliography}

\end{document}
